% Permission is granted to copy, distribute and/or modify this document
% under the terms of the GNU Free Documentation License, Version 1.3
% or any later version published by the Free Software Foundation;
% with no Invariant Sections, no Front-Cover Texts, and no Back-Cover Texts.
% A copy of the license is included in the section entitled "GNU
% Free Documentation License".
%
% Authors:
% Caner Candan <caner@candan.fr>, http://caner.candan.fr

\begin{document}

\pagestyle{empty}

\noindent
{\Large \textsc{Caner Candan}}

\noindent
\begin{flushright}
\smallskip
\lang{birth date: 11$^{th}$ september 1986}{date de naissance: 11 septembre 1986}
\hfill
phone: \lang{+33.}{}0672966884\\
\lang{citizenship: european}{citoyenneté: européenne}
\hfill
e-mail: caner@candan.fr\\
website: \url{http://caner.candan.fr}\\
\end{flushright}

\smallskip

\specialisation{\lang{Research student in computer science}{Etudiant chercheur en informatique}}

\section{\lang{Research fields}{Domaine de recherche}}

%\noindent
{\it
\lang{Parallelization and programmation models}{Parallélisation et modèles de programmation},
%\lang{decision science}{aide à la décision},
\lang{stochastic process}{systèmes stochastiques},
\lang{artificial intelligence}{intelligence artificielle},
\lang{optimization}{optimisation},
\lang{meta-heuristics}{métaheuristiques},
\lang{experimental analysis}{analyse expérimentale}.
%\lang{operations research}{recherche opérationnelle}
}

%\noindent
% DAE
\lang{%
  Parallelization of an algorithm of planification Divide-and-Evolve, linear speedup in function of the number of cores, complexity of algorithmic, thread-safe variables, reentrant functions and implementation of an interface of parallelization in the framework EO.%
}{%
  Parallélisation d'un algorithme de planification Divide-and-Evolve, accélération linéaire en fonction d'une nombre de coeurs, complexité algorithmique, variables thread-safe, fonctions réentrant et implémentation d'une interface de parallélisation dans le framework EO.%
}
%
% EDA-SA
\lang{%
  Algorithms hybridation of estimation of distribution and simulated annealing and realization and implementation of the framework EDO manipulating a distribution of probability.%
}{%
  Hybridation des algorithmes à estimation de distribution et de recuit simulé et réalisation et implémentation du framework EDO manipulant une distribution de probabilité.%
}
%
% BOPO
\lang{%
  Implementation of an automatic setting parameters interface allowing to choose some criterias and resulting a performance front.%
}{%
  Implémentation d'une interface de paramétrage automatique permettant de choisir des critères et résultant d'un front de performance.%
}
%

\section{\lang{Experience}{Expériences professionnelles}}
\begin{CV}[2]{-4.5ex}
\smallskip

% September 2009 - August 2011
\item[2009-2011]
  \textbf{%
    \lang{%
      Part-time job and MSc's internship,%
    }{%
      Travail à mi-temps et stage de Master,%
    }%
  }
  \lang{%
    THALES Research \& Technology, decision technologies and mathematics laboratory.%
  }{%
    Centre de recherche THALES, laboratoire mathématiques et techniques de la décision.%
  }

  \textit{%
    \lang{%
      Meta-parameter tuning for mono-objective and implementation by a framework C++ ``Evolving Objects''. Testing and checking for some correlations. Work on a Linux clustered system.%
    }{%
      Adaptation d’une méthode de paramétrage existante au multi-objectif et sa mise en oeuvre via le framework C++ open-source ``Evolving Objects''.  Jeu de test et détermination des corrélations vérifiées. Travail sur un cluster sous Linux.%
    }%
  }

% \myecvitemsep{\lang{September 2009 - August 2011}{Septembre 2009 - Août 2011}}{
%   \lang{
%     \textbf{THALES Research \& Technology France}, Paris (Professional training contract)\\&\textit{C++ developer, parameter of metaheuristics}\\&Meta-parameter tuning for mono-objective and implementation by a framework C++ ``Evolving Objects''. Testing and checking for some correlations. Work on a Linux clustered system. Publications\\& % needed to avoid an error
%   }{
%     \textbf{THALES Research \& Technology France}, Paris (Contrat de professionnalisation)\\&\textit{Développeur C++, paramétrage de méta-heuristiques}\\&Adaptation d’une méthode de paramétrage existante au multi-objectif et sa mise en oeuvre via le framework C++ open-source ``Evolving Objects''.  Jeu de test et détermination des corrélations vérifiées. Travail sur un cluster sous Linux.\\&Publications
%   }
% }

% January - August 2009
\item[2009]
  \textbf{%
    \lang{%
      Project manager and administrator (consultant),%
    }{%
      Chef de projet et administrateur (consultant),%
    }%
  }
  \lang{%
    Rotana, online services and digital archiving departments, Cairo, Egypt.%
  }{%
    Rotana, départements des services en ligne et d'archivage numérique, Le Caire, Égypte.%
  }

  \textit{%
    \lang{%
      Worked with two teams from Future-brand, London and Link.Net, Cairo in order to implement a new streaming technology connected to the large media database of the group: planned a road-map, developed and bug tracked some software features, wrote documentations, followed up meetings. Maintained the datacenter and developed a rights media manager for the digital arching department.%
    }{%
      Travaillé avec deux équipes de Future-brand, Londre et Link.Net, Le Caire afin d'implémenter une nouvelle technologie de streaming connectée à l'importante base de données de média du groupe: planifié la feuille de route, développé et débugué des fonctionnalités logiciel, rédigé des documents, meetings. Développé une application de contrôle d'accès aux médias pour le département d'archivage numérique.%
    }%
  }

% \myecvitemsep{\lang{January - August 2009}{Janvier - Août 2009}}{
%   \lang{
%     \textbf{Rotana}, Cairo, Egypt (Consultant)\\&\textit{Project Manager Online Services and Digital Archiving Department Administrator}\\&Follow several projects. Plan, conception, development, bug tracking and test. Writing the project's documentation. Developing of a rights manager for the Digital Arching Department. Follow up meeting suggested, bug fixing and testing. Maintaining the media network (SAN storage solution + Tape's Library) and troubleshooting. Teaching some guys using the media solutions.\\& % needed to avoid an error
%   }{
%     \textbf{Rotana}, Le Caire, Égypte (Consultant)\\&\textit{Chef de projet pour les services en ligne et administrateur pour le département d'archivage numérique}\\&Suivi de plusieurs projets. Planification, conception, développement, débogues et tests. Rédaction de documentation de projet. Développement d'une application de contrôle de droits pour le département d'archivage numérique.
%   }
% }

% % \myecvitemsep{Juillet 2007 - Janvier 2008}{\textbf{Piwi-Est}, Sarrebourg (Consultant)\\&\textit{Développeur web}\\&Xnov, gestionnaire de cabinets de chirurgiens à l'échelle nationale.}

% % \myecvitemsep{Août 2005 - Juillet 2007}{\textbf{TDE-Informatique}, Mulhouse (Alternance)\\&\textit{Développeur web}\\&Réalisation de plusieurs sites web et administration du parc informatique.}

\end{CV}

\section{\lang{Education}{Formation universitaire}}
\begin{CV}[2]{-4.5ex}
\smallskip

\item[2010-2011]
  \textbf{%
    \lang{%
      Master's degree in high performance computing,
    }{%
      Master 2 en calcul haute performance,
    }%
  }
  \lang{%
    UVSQ, ECP and ENS universities (in progress).%
  }{%
    universités UVSQ, ECP et ENS (en cours).%
  }%

\item[2009-2011]
  \textbf{%
    \lang{%
      Master's degree in information technology expert,
    }{%
      Master expert en technologie de l’information,
    }%
  }
  \lang{%
    Epitech (in progress).%
  }{%
    Epitech (en cours).%
  }%

\item[2007-2008]
  \textbf{%
    \lang{%
      Bachelor's degree in information technology expert,
    }{%
      Bachelor expert en technologie de l’information,
    }%
  }
  \lang{%
    Epitech.%
  }{%
    Epitech.%
  }%

\item[2005-2007]
  \textbf{%
    \lang{%
      Bac+2 in computer science technical,
    }{%
      Bac+2 en informatique de gestion,
    }%
  }
  \lang{%
    Strasbourg (work-based learning).%
  }{%
    Strasbourg (alternance).%
  }%

% \myecvitemsep{2005 - 2007}{
%   \textbf{Bac+2},
%   \lang{
%     Computer science technical degree, on a work-based learning, Strasbourg, France
%   }{
%     Informatique de gestion, Strasbourg
%   }
% }

\end{CV}

\section{\lang{Publications summary}{Bilan des publications}}
\lang{%
  An article in an international review and an article in a national review. A conference in a national congress. List of the main publications available on%
}{
  Un article dans une revue internationale et un article dans une revue nationale. Une conférence dans un congrès national. Liste des principales publications dispinible sur le web:%
}
\url{http://caner.candan.fr/publications}

% \ecvsection{Publications}

% \myecvitemsep{\lang{February 2011}{Février 2011}}{Parallel Divide-and-Evolve: Experiments with OpenMP on a Multicore Machine, GECCO'2011}
% \myecvitemsep{\lang{October 2010}{Octobre 2010}}{Hybride d'algorithme à estimation de distribution et de recuit simulé pour l'optimisation continue, ROADEF 2011}
% \myecvitemsep{\lang{October 2009}{Octobre 2009}}{Parallélisation de la recherche de fronts de performances pour le paramétrage de métaheuristiques, ROADEF 2010}

% %\ecvsection{Conferencies}

% \ecvsection{\lang{Certificates and Training}{Certifications et Formations}}

% \lang{
%   \ecvitem{8th to 11th June 2009}{Ardendo Advanced Technical Certification, Stockholm, Sweden}
% }{
%   \ecvitem{8 au 11 juin 2009}{Ardendo Advanced Technical Certification, Stockholm, Suède}
% }

% \lang{
%   \ecvitem{18th to 21st May 2009}{IBM DB2 UDB Administration for Linux Course, Cairo, Egypt}
% }{
%   \ecvitem{18 au 21 mai 2009}{Formation IBM DB2 UDB Administration for Linux, Le Caire, Égypte}
% }

% \lang{
%   \ecvitem{12th to 16th April 2009}{IBM Tivoli Storage Manager 5.3 Implementation Course, Cairo, Egypt}
% }{
%   \ecvitem{12 au 16 avril 2009}{Formation IBM Tivoli Storage Manager 5.3 Implementation, Le Caire, Égypte}
% }

% \lang{
%   \ecvitem{2nd August 2008}{TOEIC Certificate, Strasbourg, France}
% }{
%   \ecvitem{2 août 2008}{Certification TOEIC, Strasbourg}
% }

% \ecvsection{\lang{Skills}{Compétences et intérêts personnels}}

% \myecvsubsection{\lang{Advanced knowledge}{Connaissances avancées}}

% \ecvitem{\lang{Main languages}{Langages principaux}}{Assembler X86, C99, C++, Objective-C, J2EE}
% \ecvitem{\lang{Object-oriented programming}{Programmation orientée objet}}{\lang{Knowledge in design patterns and anti-patterns}{Connaissance en patron de conception et antipattern}}
% \ecvitem{\lang{Domain-specific language}{Langages dédiés (DSL)}}{\lang{Parser}{Analyseurs} Lex, Yacc, Codeworker}
% \ecvitem{\lang{Graphical Application development}{Application graphique}}{Gtk, WxWidgets, Qt, Ncurses}
% \ecvitem{\lang{Application development 2/3D}{Application 2/3D}}{SDL, Irrlicht, Ogre}
% \ecvitem{\lang{Web development}{Application Web et interprétée}}{PHP, Perl, ASP, JavaScript/Ajax, Ruby, Python, Xslt}
% \ecvitem{\lang{Databases}{Bases de données}}{Oracle, DB2, PostgreSQL, MySQL, SQLite}
% \ecvitem{\lang{Parallelization and Optimization}{Parallélisation et Optimisation}}{CUDA, MPI, OpenMP, ParadisEO}

% \\

% \myecvsubsection{\lang{Other knowledges}{Autres connaissances}}{
%   \lang{Embedded}{Système embarqué} (AVR)\\&
%   \lang{Brain-computer interface}{Interface neuronale directe (BCI)}\\&
%   \lang{Optical character recognition}{Reconnaissance optique de caractère} (Tessaract, OCRopus, Conjecture)\\&
%   \lang{Segmentation and recognition}{Segmentation d'image} (OpenCV)
% }

% \myecvsubsection{\lang{Softwares and workstations}{Logiciels et environnements de travail}}

% \ecvitem{\lang{Operating System}{Environnements}}{*BSD, GNU/Linux, Solaris}
% \ecvitem{\lang{Editors}{Éditeurs}}{Emacs, Eclipse, Netbeans}
% \ecvitem{\lang{Compute and Modelize}{Calcul et Modélisation}}{GNU Octave, Scilab, Gnuplot}
% \ecvitem{\lang{SCM}{Contrôleurs de codes}}{Subversion, CVS, Bazaar, Mercurial, Git}
% \ecvitem{Bug tracker}{Trac, FlySpray, Bugzilla}
% \ecvitem{\lang{Automatic programming}{Atelier de génie logiciel}}{ArgoUML, KDevelop, StarUML, Clarion, Dia, Umbrello}
% \ecvitem{\lang{Software documentation}{Documentation logicielle}}{Doxygen, JavaDoc, LaTex, DokuWiki}

% \ecvsection{\lang{Languages}{Langues étrangères}}

% \lang{
%   \ecvitem{English}{good knowledge, TOEIC August 2008 (scoring 790)}
%   \ecvitem{Turkish and French}{native speaker}
%   \ecvitem{German and Arabic}{notions}
% }{
%   \ecvitem{Anglais}{lu, écrit et parlé (niveau TOEIC)}
%   \ecvitem{Turc}{langue maternelle}
%   \ecvitem{Allemand et Arabe}{notions}
% }

% \ecvsection{\lang{Interests}{Centres d'intérêts}}

% \lang{
%   \ecvitem{Sports}{karate, football, basketball and bike}
%   \ecvitem{Hobbies}{energy-efficient technologies, history, culture, foreign languages, travel, cinema, music and reading.}
% }{
%   \ecvitem{Sports}{karaté, football, basket-balls et vélo}
%   \ecvitem{Loisirs}{veille technologique, histoire, cultures et langues étrangères, voyages, cinéma, musique et lecture.}
% }

% \lang{
%   \ecvitem{\textbf{References}}{Available upon request}
% }{
%   \ecvitem{\textbf{Références}}{Disponible sur demande}
% }


%\vspace{1cm}

\end{document}
