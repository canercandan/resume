% Permission is granted to copy, distribute and/or modify this document
% under the terms of the GNU Free Documentation License, Version 1.3
% or any later version published by the Free Software Foundation;
% with no Invariant Sections, no Front-Cover Texts, and no Back-Cover Texts.
% A copy of the license is included in the section entitled "GNU
% Free Documentation License".
%
% Authors:
% Caner Candan <caner@candan.fr>, http://caner.candan.fr

\begin{document}

\pagestyle{empty}

\noindent
{\Large \textsc{Caner Candan}}

\noindent
\begin{flushright}
\smallskip
\lang{birth date: 11$^{th}$ september 1986}{date de naissance: 11 septembre 1986}
\hfill
phone: \lang{+33.}{}0672966884\\
\lang{citizenship: french}{citoyenneté: français}
\hfill
e-mail: caner@candan.fr\\
website: \url{http://caner.candan.fr}\\
\end{flushright}

\smallskip

\specialisation{\lang{PhD student in computer science}{Doctorant en informatique}}

\bigskip

% Research fields
%\section{\lang{Research fields}{Domaine de recherche}}

{\it
\lang{Parallelization and programmation models}{Parallélisation et modèles de programmation},
%\lang{decision science}{aide à la décision},
\lang{stochastic optimization algorithms}{algorithmes d'optimisation stochastique},
\lang{artificial intelligence}{intelligence artificielle},
\lang{optimization}{optimisation},
\lang{meta-heuristics}{métaheuristiques},
\lang{experimental analysis}{analyse expérimentale},
\lang{careful programming approach}{approche de programmation soigneuse}.
%\lang{operations research}{recherche opérationnelle}
}

\bigskip

{\bf \lang{Main achievements}{Réalisations principales}:}
%
% DIM
\lang{%
  Parallelization of an autonomous island models algorithm implemented as a C++ framework in both MPI and SMP paradigms.%
}{%
}
%
% Asynchronous DIM
\lang{%
  Conception and implementation of a new asynchronious and non-blocking approach of the Dynamic Islands Model.
}{%
}
%
% DAE
\lang{%
  Parallelization of an evolutionary algorithm, linear speedup in function of the number of cores, thread-safe variables and reentrant functions.%
}{%
  Parallélisation d'un algorithme évolutionniste, accélération linéaire en fonction du nombre de coeurs, variables thread-safe et fonctions réentrantes.%
}
%
% EDA-SA
\lang{%
  Hybridation of estimation of distribution algorithm and simulated annealing.%
}{%
  Hybridation d'un algorithme d'estimation de distribution et de recuit simulé.%
}
%
%% % BOPO
%% \lang{%
%%   Implementation of an automated parameter setting interface allowing to choose some criteria and searching for a performance front.%
%% }{%
%%   Implémentation d'une interface de paramétrage automatique permettant de choisir des critères et de chercher un front de performance.%
%% }
%% %

% Publications
\lang{%
  {\bf Publications summary:} An article in an international congress and an article in a national congress. List of publications available at%
}{
  {\bf Bilan des publications:} Un article dans un congrès international et un article dans un congrès national. Liste des publications disponible sur le web:%
}
\url{http://caner.candan.fr/Publications/List}

% Experience
\section{\lang{Experience}{Expériences professionnelles}}
\begin{CV}[2]{-4.5ex}
\smallskip

% September 2011 - August 2014
\item[2011-2013]
  \lang{%
    {\bf PhD researcher,} MOA (Metaheuristics, Optimization and Applications) team, LERIA (Computer Science Laboratory), University of Angers.

    {\it Parallelization of an autonomous island models algorithm for selection of operators using EO framework. Efficiency assessed on the One-Max and TSP problems. New asynchronious and non-blocking approach for the islands model.}
  }{%
  }%

% September 2009 - August 2011
\item[2009-2011]
  \lang{%
    {\bf Part-time job and MSc's internship,}
    THALES Research \& Technology, decision technologies and mathematics lab.

    {\it Parallelization of a planning algorithm using EO framework. Hybridation of EDA \& SA. Implementation of an automated parameter setting interface.}%
  }{%
    {\bf Travail à mi-temps et stage de Master,}
    Centre de recherche THALES, laboratoire mathématiques et techniques de la décision.

    {\it Parallélisation d'un algorithme de planification utilisant le framework EO. Hybridation de EDA-SA. Implémentation d'une interface de paramétrage automatique.}%
  }%

% January - August 2009
%% \item[2009]
%%   \lang{%
%%     {\bf Project manager and administrator (consultant),}
%%     Rotana, online services and digital archiving departments, Cairo, Egypt.

%%     {\it In collaboration with the teams of Future-brand, London and Link.Net, Cairo to implement a new streaming technology connected to the media database: planning a road-map, developing software features, writing documentations, following up meetings. Maintaining the datacenter.}%
%%   }{%
%%     {\bf Chef de projet et administrateur (consultant),}
%%     Rotana, services en ligne et archivage numérique, Le Caire, Égypte.

%%     {\it En collaboration avec les équipes de Future-brand, Londre et Link.Net, Le Caire pour implémenter une nouvelle technologie de streaming connectée à la base de données: planification de la feuille de route, développement des fonctionnalités logiciel, rédaction de documents, meetings. Maintien du datacenter.}%
%%   }

\end{CV}

% Education
\section{\lang{Education}{Formation universitaire}}
\begin{CV}[2]{-4.5ex}
\smallskip

\item[2011-2014]
  \lang{%
    {\bf PhD degree in Autonomous Island Models for Combinatorial Optimisation,} University of Angers.
  }{%
    {\bf PhD degree in Modèles en iles autonomes pour l'optimisation combinatoire,} University of Angers.
  }%

\item[2010-2011]
  \lang{%
    {\bf Master of research degree in high performance computing,}
    University of Versailles Saint-Quentin-en-Yvelines, École Centrale Paris and École Normale Supérieure Cachan.%
  }{%
    {\bf Master 2 recherche en calcul haute performance,}
    Université de Versailles Saint-Quentin-en-Yvelines, École Centrale Paris et École Normale Supérieure Cachan.%
  }%

\item[2009-2011]
  \lang{%
    {\bf Master of engineering degree in information technology expert,} Epitech, Paris.%
  }{%
    {\bf Master ingénieur, expert en technologie de l’information,} Epitech, Paris.%
  }%

%% \item[2007-2008]
%%   \lang{%
%%     {\bf Bachelor's degree in information technology expert,} Epitech, Strasbourg.%
%%   }{%
%%     {\bf Bachelor expert en technologie de l’information,} Epitech, Strasbourg.%
%%   }%

%% \item[2005-2007]
%%   \lang{%
%%     {\bf Higher national diploma in management information system,} Strasbourg (work-based learning).%
%%   }{%
%%     {\bf Bac+2 en informatique de gestion,} Strasbourg (alternance).%
%%   }%

\end{CV}

% Achievements
\section{\lang{Achievements}{Réalisations}}
\begin{CV}[2]{-4.5ex}
\smallskip

\item[2013]
  \lang{%
    {\bf Conception} and implementation of an asynchronious and non-blocking island models algorithm. Several levels of asynchronous communications were defined among the migration processes and feedback information sharing. Communications are made thanks to the message passing interface (MPI) and multi-threaded operations layers (SMP). Available at \url{http://forge.info.univ-angers.fr/redmine/projects/dim}
  }{%
  }%

\item[2012]
  \lang{%
    {\bf Conception} and implementation of a generic framework for the Dynamic Islands Model (DIM) in a higher layer above the framework EO ``Evolving Objects'' (C++). Both the MPI and SMP parallelization paradigms of the DIM were implemented as well. Available at \url{http://forge.info.univ-angers.fr/redmine/projects/dim} (Published at GECCO 2012)
  }{%
  }%

\item[2011]
  \lang{%
    {\bf Implementation} of a parallelization module in the framework EO, using OpenMP. Scalability issues in variation and replacement operators due mainly to memory allocations not under control. Works with larger population sizes. Experiments, on a 48-core machine, showed significant speedups with an increasing number of cores. Available at \url{http://eodev.sf.net} (Published at GECCO 2011)
  }{%
    {\bf Implémentation} d'un module de parallélisation dans le framework EO ``Evolving Objects'' en C++, utilisant OpenMP. Problèmes de passage à l'échelle dans les opérateurs de variation et remplacement dûs principalement aux allocations mémoires non maîtrisées. Fonctionne avec des tailles de population plus grandes. Les expériences, sur une machine à 48 coeurs, ont montrées des accélérations sinificatives en fonction du nombre de coeurs. Disponible sur le web: \url{http://eodev.sf.net} (Publication à GECCO 2011).
  }%

%% \item[2011]
%%   \lang{%
%%     {\bf Conception} and implementation of a linear algebra computation framework in C++ along with a layout abstraction for parallelization paradigms. Provides operators to compute dense and sparse matrices with generically designed scalar, complex, vector and matrix types. Available at \url{https://github.com/canercandan/linear-algebra}
%%   }{%
%%     {\bf Conception} et implémentation d'un framework de calcul d'algèbre linéaire en C++ avec un niveau d'abstraction pour les paradigmes de parallélisation. Le framework fournit des opérateurs de calcul sur matrices denses et creuses avec des types scalaires, complexes, vecteurs et matrices génériques. Disponible sur le web: \url{https://github.com/canercandan/linear-algebra}
%%   }%

\item[2010]
  \lang{%
    {\bf Conception} and implementation of the EDO framework, for the design of estimation of distribution algorithms. Used along with ParadisEO-MO to implement an original hybridation of multi-variate gaussian EDA with simulated annealing. Available at \url{http://eodev.sf.net} (Published at ROADEF 2011)
  }{%
    {\bf Conception} et implémentation du framework EDO, pour le design d'algorithmes à estimation de distribution. Utilisé avec ParadisEO-MO pour implémenter une hybridation originale de gaussienne multi-variante EDA avec un recuit simulé. Disponible sur le web: \url{http://eodev.sf.net} (Publication à ROADEF 2011)
  }%

%% \item[2009]
%%   \lang{%
%%     {\bf Implementation} of a multi-objective parameter setting and behaviour extraction method, using ParadisEO framework.
%%   }{%
%%     {\bf Implémentation} d'une méthode de paramétrage multi-objectifs et d'extraction de comportements utilisant le framework ParadisEO.
%%   }%
%%   \url{http://paradiseo.gforge.inria.fr}

\end{CV}

\section{\lang{Computer science}{Informatique}}
\begin{CV}[2]{-4.5ex}
\smallskip

\item
  \textbf{\lang{Technical tools}{Outils techniques}:}
  *BSD, GNU/Linux, Solaris —
  Assembler X86, C99, C++, Objective-C, J2EE —
  Lex, Yacc, Codeworker —
  PHP, Perl, JavaScript/Ajax, Ruby, Python, Xslt —
  CUDA, MPI, OpenMP —
  GNU Octave, Scilab, Gnuplot —
  Subversion, Git —
  Oracle, DB2, PostgreSQL, MySQL, SQLite.

\end{CV}

\section{\lang{Interests}{Centres d'intérêts}}
\begin{CV}[2]{-4.5ex}
\smallskip

\lang{
\item \textbf{Sports:} karate, football, basketball and bike
\item \textbf{Hobbies:} energy-efficient technologies, history, culture, foreign languages, travel, cinema, music and reading.
\item Particular interest in free and open source software.
}{
\item \textbf{Sports:} karaté, football, basket-balls et vélo
\item \textbf{Loisirs:} veille technologique, histoire, cultures et langues étrangères, voyages, cinéma, musique et lecture.
\item Intérêt particulier pour les logiciels libres.
}

\end{CV}

\end{document}
